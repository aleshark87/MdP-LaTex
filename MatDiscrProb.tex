
\documentclass[11pt]{article}
\usepackage{hyperref}
\begin{document}
    \section*{\huge \centering Matematica Discreta e Probabilit\'a }
     \subsection*{Suddivisione corso} 
    
        Il corso \'e suddiviso in quattro parti: calcolo combinatorio,
        statistica descrittiva, probabilit\'a e statistica inferenziale. \\
        Verranno dedicate 10 ore per la parte di calcolo combinatorio, 5 
        per la parte di statistica descrittiva, 40 per la parte di probabilit\'a e 5 per la parte di statistica inferenziale.
   
   \section{Calcolo Combinatorio}
    \subsection{Richiami}
        Serviranno i concetti di \textbf{insieme},\textbf{ sottoinsieme}, e di funzioni \textbf{iniettive} e \textbf{suriettive}.
        \subsubsection{Insieme}
            Un \textbf{insieme} \'e un raggruppamento di elementi che pu\'o essere individuato mediante una caratteristica comune che gli appartengono.\\
            $A=\{a,b\}$
        \subsubsection{Sottoinsieme}
            Un \textbf{sottoinsieme} \'e un raggruppamento di elementi appartenenti ad un insieme che \'e anch'esso un insieme.\\
            $B=\{n \in N : 1\leq n \leq 5 \} $ 
        \subsubsection{Funzioni Iniettive}
            Una funzione si dice \textbf{Iniettiva} se elementi distinti del dominio hanno immagini distinte. Possono esistere elementi del codominio non raggiunto da elementi del dominio.
        \subsubsection{Funzioni Suriettive}
            Una funzione si dice \textbf{Suriettiva} se ogni elemento del codominio \'e raggiunto da almeno un elemento del dominio. Non possono esistere elementi del codominio non raggiunto da elementi del dominio.
        \subsubsection{Funzioni Biettive}
            Una funzione si dice \textbf{Biettiva} se \'e sia suriettiva che iniettiva.
    \subsection{Teoria della Cardinalit\'a}
        Due insiemi hanno la stessa \textbf{cardinalit\'a} se esiste una biezione tra essi.\\
        Se \'e possibile costruire una funzione biettiva tra due insiemi.
        \subsubsection{Definizioni}
            Un insieme $A$ si dice finito se esiste $n \geq 0$ tale che $A$ e $\{1,2,..n\}$ hanno la stessa \textbf{cardinalit\'a}.  \\
            \\
            Se $A \longleftrightarrow \{1,..,n\}$ diciamo che $A$ ha \textbf{cardinalit\'a} $n$ e scriviamo  
            $\mid A \mid  =  n $ 
            \\
            \\
            Un insieme $A$ si dice \textbf{numerabile} se esiste biezione. 
            $A \longleftrightarrow N = \{0,1,2,..\}$
        \subsubsection{Esempi}
            \textbf{\large 1}\\
            $B=\{1,2,3,..\} \ \ f:N\to B , f(n)=n+1 $ \\
            La funzione rappresenta la corrispondenza tra $B$ e l'insieme dei numeri interi positivi (infiniti) $\mapsto$ $B$ \'e \textbf{numerabile}.\\
            \textbf{\large 2}\\
            $Z=\{0,1,-1,2,-2,..\}$ \'e \textbf{numerabile}? \\
            Si, lo \'e perch\'e basta fare le giuste associazioni : \\
            $0\mapsto 0 $, $1\mapsto1$, $-1\mapsto2$, $2\mapsto3$\\
            $f:Z\to N$ $\{2z-1 $ se $z>0$ \\
            \hspace*{1,8cm} $\{-2z  $ \hspace{0,35cm}se $z\leq0$  \\
            \textbf{Osservazione}\\
            Un insieme \'e \textbf{numerabile} se riesco ad elencare tutti i suoi elementi in un elenco infinito e viceversa.\\
            \textbf{3}\\
            $Q \geq 0 =\{\frac{a}{b}:a,b \in N , b\ne 0\}$ (Tutte le frazioni $>$ 0)\\
            \'E \textbf{numerabile}, l'importante \'e poter esprimere le frazioni in un elenco.\\
            $0,1,\frac{1}{2},2,\frac{1}{3},\frac{2}{3},\frac{3}{2},3,\frac{1}{4},\frac{3}{4},\frac{4}{3},4, ...$\\ In questo modo posso sicuramente scrivere \textbf{tutti} i numeri razionali.\\
            \\\\\\
            \textbf{\large 4}\\
            $A=\{$sequenze binarie di lunghezza infinita$\}$\\
            Vogliamo mostrare che $A$ non \'e \textbf{numerabile}. Prendiamo un elenco infinito di elementi di A.\\
            $a_{1}$ $=$ $0110101..$ , $a_{2}$ $=$ $1100101..$ , $a_{3}$ $=$ $0010100..$ , $...$ , $a_{n}$ $=$ $1000011..$\\
            Vogliamo costruire una sequenza che non fa parte di questo elenco. \\
            Come primo bit mettiamo il bit \textbf{opposto} a quello del primo di $a_{1}$.\\
            Come secondo bit mettiamo il bit \textbf{opposto} a quello del secondo di $a_{2}$.\\
            Come terzo bit mettiamo il bit \textbf{opposto} a quello del terzo di $a_{3}$.\\
            Ragionando in questo modo fino ad n avremo una sequenza che non fa parte di $A$.
            \\$A$ non \'e \textbf{numerabile} , perch\'e \'e impossibile creare un elenco infinito che contenga tutte le sequenze di $A$.\\
            \textbf{Osservazione}\\ 
            Questo insieme $A$ e $R$ hanno la stessa cardinalit\'a. Tale cardinalit\'a si dice \textbf{del continuo}. \url{http://www.scit.wlv.ac.uk/~cm1993/maths/mm2217/rup.htm}
        \subsubsection{Esercizi}
            \textbf{\large 1}\\
            Abbiamo due biglietti da regalare a due amici scelti su 10. In quanti modi posso fare la scelta?
            \\
            \textbf{\large Soluzione}\\
            Possiamo partire puntando la prima persona ($1$) e contando con quante possibili persone la persona $1$ può andare al concerto. Siccome sono $10$ persone, allora la persona $1$ potrà andare al concerto con $9$ persone.
            Ora controlliamo con quante persone la persona $2$ può andare al concerto (escludendo l'esistenza della persona $1$), e sta volta saranno $8$.
            Ora vedremo uno schema ricorsivo per il conteggio delle combinazioni siccome la persona $3$, potrà andare al concerto con $7$ persone (escludendo questa volta la persona $1, 2$).
            Ora per contare le possibili combinazioni \`e facile vedere che ci basta fare una sommatoria da $1$ a $9$.
            \\    
            $\sum_{i=1}^{9}i = 45$, using Gauss formula $\frac{n(n+1)}{2}$.
            \\
            \\
            \textbf{\large 2}\\
            C'è un torneo di un gioco (non specificato) in quale abbiamo 3 concorrenti A, B, C. Quali sono le possibili combinazioni del podio contando anche i pareggi?
            \\
            \textbf{\large Soluzione}\\
            $1^o$ Caso: nessun pareggio. 6 possibili combinazioni:\\
            1) A,B,C\\
            2) A,C,B\\
            3) B,A,C\\
            4) B,C,A\\
            5) C,A,B\\
            6) C,B,A\\
            \\
            Possiamo trovare queste combinazioni calcolando $3! = 6$.\\
            $2^o$ Caso: tutti parimerito. $1$ Combinazione:\\
            A,B,C (Indipendente dall'ordine)\\
            $3^o$ Caso: due parimerito. $6$ Combinazioni:\\
            1) (A,B),C o alternativamente C,(A,B). () Indica stessa posizione/pareggio, siccome sono $3$ concorrenti moltiplico queste $2$ possibili combinazioni per $3$, quindi $3 \cdot 2 = 6$.\\
            Totale = $6 + 1 + 6 = 13$.\\
            \\
            \textbf{\large 3}\\
            Lancio un dado $3$ volte. Quante sono le possibili sequenze di risultati in cui in ordine di grandezza il dado con valore più alto è $\geq 5$, il secondo più alto $\geq 4$, il terzo più alto $\geq 3$.\\
            \textbf{\large Soluzione}\\
            Come prima, dividiamo il problema in sottoproblemi e studiamo caso per caso.\\
            $1^o$ Caso: $3$ risultati diversi:\\
            1) $4, 5, 6$\\
            2) $3, 4, 5$\\
            3) $3, 4, 6$\\
            4) $3, 5, 6$\\
            Ognuna di queste ha 6 possibili combinazioni.\\
            Totale temporaneo = $4 \cdot 6 = 24$.\\
            $2^o$ Caso: $2$ risultati uguali $1$ diverso:\\
            1) $4,4,5$\\
            2) $4,4,6$\\
            3) $5,5,3$\\
            4) $5,5,4$\\
            5) $5,5,6$\\
            6) $6,6,3$\\
            7) $6,6,4$\\
            8) $6,6,5$\\
            E ognuna di queste può avere $3$ combinazioni diverse, \\quindi totale temporaneo = $24 + (8 \cdot 3) = 48$.\\
            $3^o$ Caso: $3$ risultati uguali:\\
            1) $5,5,5$\\
            2) $6,6,6$\\
            Totale = $48 + 2 = 50$.\\
            Non contiamo il caso $1$ risultato uguale e $2$ diversi perchè non ha senso, siccome $1$ uguale a cosa? lol..\\
            \textbf{\large 4}\\
            L'insieme $A = \{1,2,...,11\}$ ha più sottoinsiemi di cardinalità pari o dispari o il numero di sottoinsiemi pari è uguale a quello di dimensione dispari?\\
            \textbf{\large Soluzione}\\
            Possiamo usare di ciascun sottoinsieme il suo complementare, esempio con $n = 3$ : \\
            $\{1,2\} \longleftrightarrow 3$\\
            $\{1,3\} \longleftrightarrow 2$\\
            $\{2,3\} \longleftrightarrow 1$\\
            $\emptyset \longleftrightarrow \{1,2,3\}$\\
            Per induzione quindi avremo sempre degli insiemi pari complementari a quelli dispari e quindi il numero di sottoinsiemi pari è uguale a quelli dispari.\\
            \textbf{\large 5}\\
            4 coppie moglie e marito, sono scambisti per una serata, in quanti modi possono scambiarsi? (Sono etero).\\
            \textbf{\large Soluzione}\\
            Si possono scambiare in 9 modi diversi (Risposta da chiarire mercoledi).
            \\
            \subsection{Proprietà Cartesiane}
            \subsubsection{Prodotto Cartesiano}
            $A,B$ insiemi.\\
            $A \times B = \{(a,b) : a \in A, b \in B\}$\\
            Si estende in moto naturale al caso di $3$ o più insiemi.\\
            \\
            \textbf{\large Esempio}\\
            $A = \{1,2\}$ e $B = \{1,a,3\}$. Allora :
            $A \times B = \{(1,1), (1,a), (1,3), (2,1),(2,a),(2,3)\}$\\
            \subsubsection{Potenza Cartesiana}
            $A^n = A \times A \times A \times ... \times A$ ($n$) volte.\\
            $A^n$ è l'insieme dell n-ple di ordinate di elementi di $A$.
            Gli elementi di $A^n$ si dicono \textbf{liste} o \textbf{sequenze}.\\
            \\
            \textbf{\large Esempio}\\
            $A = \{0,1\}$, allora $A^3 = \{(0,0,0), (0,0,1),...,(1,1,1) \}$.\\
            \subsection{L'insieme delle parti}
            $A$ insieme, l'insieme delle parti di $A$, è l'insieme $P(A)$ i cui elementi sono i sottoinsiemi di $A$.\\
            $A = \{1,a,3\}$.\\
            $P(A) = \{\emptyset, A, \{1\}, \{a\}, \{3\}, \{1,a\}, \{1,3\}, \{a,3\}\}$, $|P(A)| = 8$.\\
            \subsubsection{Definizione}
            Sia $A$ un insieme e $|A| = n$. ALlora esiste una biiezione tra $P(A)$ e ${0,1}^n$.
            \subsubsection{Dimostrazione}
            Iniziamo con un esempio $A = \{1,2,3\}$.\\
            $\{\emptyset\} \longrightarrow 000$\\
            $\{1,2,3\} \longrightarrow 111$\\
            $\{1\} \longrightarrow 100$\\
            $\{2\} \longrightarrow 010$\\
            $\{3\} \longrightarrow 001$\\
            $\{1,2\} \longrightarrow 110$\\
            $\{1,3\} \longrightarrow 101$\\
            $\{2,3\} \longrightarrow 011$\\
            In generale se $A = \{1,2,...,n\}$ allora $F:P(A) \to \{0,1\}^n$.\\
            Sia $S \subseteq A$, $F(S) = (b_1,b_2,...,b_n) \in \{0,1\}^n$ con $\begin{cases}
            1 if a_i \in S\\
            2 if a_i \not\in S\\
            \end{cases}$
            Da aggiustare.\\
            
                
\end{document}
